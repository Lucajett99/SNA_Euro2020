\documentclass[12pt, a4paper]{article}
\usepackage[utf8]{inputenc}
\usepackage[english]{babel}
\usepackage{float}
\usepackage{newlfont}
\usepackage{hyperref}
\usepackage{multirow}
\usepackage{subfiles}
\usepackage{graphicx}
\usepackage{subfig}
\usepackage[toc,page]{appendix}

\textwidth=450pt\oddsidemargin=0pt
\graphicspath{ {./images/} }
\raggedbottom


\author{Marco Benito Tomasone 1038815 \\ 
        Luca Genova  1038843\\
 Master Degree in Computer Science\\}
\date{2022-2023}
\title{Network Analysis:\\ Network Structure and Team Performance: Euro 2020}

\begin{document}
\maketitle

\section{Abstract}
\label{abstract}
In this work we analyze the network structure of football teams and how it affects the performance of the team. We focused this work on the last UEFA Euro 2020 tournament. Analyzing the network created from the passes of the player we showed that increasing the intensity and decreasing the centralization of a network leads to better performance. We applied also a qualitative analysis which has the purpose to show how analyzing the network structure of a team could be useful for improving the team performance and to apply countermeasures to opponent's style of play.\\

\section{Context}
\label{context}
Human interaction are very important in our society. In a lot of situations humans tend to group themselves in subsets to operate in a more efficient way. Also in sports, humans organized themselves in teams to achieve common goals and demonstrate their skills and their ability to work in a group. In this case of interaction, the ways in which the members of the team interact with each other are more important, because the way people inside the team interact influence directly the results the team will obtain. In this study we will analyze the network structure of football teams and how it affects the performance of the team. We focused this work on the last UEFA Euro 2020 tournament, where UEFA members' senior men's national teams face each other determining the continental champion of Europe.
\subsection{Previous Work}
This work is heavily based on the work done by Thomas U. Grund \cite{GRUND} in which he analyzed two seasons of the English Premier League. He analyzed 1520 networks and tested two hypotheses we want to test in our work. 
%%%%%%%%
%The context includes: the general field (e.g., literature, history,
%archaeology, tourism, biology, forensics, religious studies); the
%specific application (e.g., literary analysis, quantitative history,
%genetics, virology, forensics intelligence, tourism planning, biblical
%quantitative studies).

\section{Problem and Motivation}
\label{problem-and-motivation}
In last years data science is a very growing field and this it's impacting a lot of different fields, sport included. In the football case, data are in most case an unknown technology and a lot of teams (even at highest level) neither use this type of knowledge to understand the performance of the team itself nor the opponents. Given that the way players interact each other can result in a better or worse performance of the team, it's important to understand how the network structure of the team can affect the performance of the team. Our work aims to show how analyze the data in football can be a leading change for the sport itself and how important analyzing the network structure of a team from a quantitative and a qualitative point of view can be. Our main contribution will be to demonstrate that the hypothesis proposed and tested by Grund \cite{GRUND} are also valid for a different context. In fact, an international tournament is very different from a domestic league because in this type of tournament the level is higher and the every match can be decisive for the final result, so the importance of every single detail is higher. 

%%%%%%%%%%%%%
%What are the problems you want to address? Why are those problems
%important (impact, theoretical and/ or practical needs, etc.)? What are
%the main contributions of the project?

\section{Datasets}
\label{datasets}
The data have been provided by StatsBomb, one the biggest provider for football data. The dataset is composed by every single event during the match: every pass, every shoot, every press and so on. 
The gathering process is based on a Python library called StatsBombPy, which is a wrapper for the StatsBomb API. 
StatsBomb's Data are not for free, except for some free data: Euro 2020 is one of them. From StatsBombPy you get data in form of a Pandas Dataframe, which is a very useful tool for data analysis. We used Python and Pandas (a Python library) to manage data, then we gathered data and store data for all matches in .csv or .xlsx format. All other manipulation on data have been done using python and metrics have been computed using Python and  NetworkX (a Python library for network analysis). The data we gathered contains all the matches of the tournament, so we have data for 51 matches. In a football match there are two team facing, so we have a total of 102 networks. We computed metrics for each network and then we analyzed the results.
%%%%%%%%%%%%%
%How did you gather the data? Did you digitise it? How? Is the material
%publicly available? What tools did you use 1) to handle (store,
%manipulate) the data and 2) to compute measures on the data?

\section{Validity and Reliability }
\label{validity-and-reliability}
 
%%%%%%%%%%%%%%%%
% How closely does the model of your dataset represent reality (validity)?
% How consistent is the model you assembled (reliability)?


\section{Hypothesis}
Our work is based on two fundamental hypothesis:
\begin{enumerate}
        \item Team performance is affected by interaction opportunities: increased interaction intensity leads to increased team performance.
        \item Increased centralization of interaction in teams leads to decreased team performance.
\end{enumerate}
So the first hypothesis is based on the idea that the more a team interacts with each other, the more the team will be able to perform better. The second hypothesis is based on the idea that the more a team is centralized on a single player, the less the team will be able to perform better, because if a team makes to much reliance on a single player (or few players), the team will be more vulnerable to the opponent. 
\section{Measure}
\label{measures}
Based on the hypothesis we want to prove we focused on two metrics in particular:
\begin{itemize}
        \item Network Intensity
        \item Network Centralization
\end{itemize}
\subsection{Network Intensity}
Usually one of the most computed metrics of networks is density, which
is traditionally calculated as the number of existing ties in a network divided by the number of potential ties. In our case, we can't use this metric, because in a football match we can easily expect a pass between each pair of players in a team. So we need to define a new metric, which is weighted on the number of passes. To compute this new metric we need the information of the number of passes received and made by each player. So we used: 
\begin{itemize}
        \item \textbf{Out-strenght:} the number of passes made by the player $i$
        $$ C_{OS}(i) = \sum^{N}_{j=1}w_{ij}$$
        \item \textbf{In-strenght:} the number of passes received by the player $i$
         $$C_{IS}(i) = \sum^{N}_{j=1}w_{ji}$$
\end{itemize}
Where N is the number of nodes in the network, in our case is 11, because we consider only the starting XI of each team. We focused our attention on the starting XI instead of the best eight player for number of passes (as in Grund \cite{GRUND}) because we are analyzing a tournament. In a tournament in the final phase if two teams draw go to extra time so some player from the bench could impact the match in a more important way than others. To limit this problem we only focused on the starting XI. Another reason is that these tournament are very short (7 match at maximum for a team) and the matches are knockout game, so all the coach try to always play with the best players he has for that match. Another reason for which we focused only on the starting XI is that from 2020 (due to the pandemic) the number of subs a coach can do is 5, while in the past it was 3 (that's because Grund used only the most eight players, because he ideally consider only the players who played the entire match). \\
So the network intensity for a team is defined as:
$$I = \frac{1}{T}\sum^N_{i=1} \frac{ C_{OS}(i) +  C_{IS}(i)}{2}$$
That is just the total number of passes made by a team in a match divided by the ball possession time in minutes. Given the data we had we can compute the effective ball possession time, so the sum of the time of possession of both team in a match does not sum up to 90 or 120 minutes, because in a match there are also the stoppages for injuries, substitutions, goals and so on. This can explain why the range of values the network intensity assumes in our case is slightly different from the one in Grund's paper.
\\
\subsection{Network Centralization}
The network centralization is computed by computing two different metrics: the weighted centralization and the strenght centralization. \\
\subsubsection{Weighted centralization}
The weighted centralization is one of the simplest way to examine the distribution of the weights in a network. It is defined as:
$$C_w = \frac{\sum^N_{i=1} \sum^N_{j=1} (w^* - w_{ij})}{(N^2 - N - 1)IT}$$
Where $w^*$ is the biggest tie value in the network (so the biggest number of passes between two players), $w_{ij}$ is the weight of the link between the node $i$ and the node $j$, $N$ is the number of nodes in the network (11) and $IT$ is the total number of passes of the team for that match. \\
This metric is zero in the  the most decentralized interaction pattern so when everybody interacts with everybody with the same intensity. In contrast, this metric is maximum (1) in the case the  most interactions involve the same two individuals. 
\subsubsection{Strenght centralization}
The streght centralization is used instead of the degree centralization, beacuse as said for the network density in (almost all) football matches we can expect a pass between each pair of players in a team. So we used a centralization for incoming and outcoming node strenght: 
$$C_I = \frac{\sum^N_{i=1}(C_{IS}^* - C_{IS}(i))}{(N - 1)IT}$$
$$C_O = \frac{\sum^N_{i=1}(C_{OS}^* - C_{OS}(i))}{(N - 1)IT}$$
Where $C_{IS}^*$ is the biggest incoming node strenght in the network (so the biggest number of passes received by a player), $C_{IS}(i)$ is the incoming node strenght of the node $i$, $C_{OS}^*$ is the biggest outcoming node strenght in the network (so the biggest number of passes made by a player), $C_{OS}(i)$ is the outcoming node strenght of the node $i$, $N$ is the number of nodes in the network (11) and $IT$ is the total number of passes of the team for that match. \\


%What measures did you apply (brief explanation of how they work)? How do
%they relate to the intent of the study? Why are they relevant?

\section{Results}
\subsection{Quantitative Analysis}
\label{quantitative-analysis}
We start the result section presenting a table summarising the results of the metrics we obtained for each of the 102 observation we made. \\
\begin{table}[H]
        \centering
        \begin{tabular}{|c|c|c|c|c|c|}
                \hline
                & \textbf{Mean} & \textbf{Std\_dev} & \textbf{Min} & \textbf{Max} &   \textbf{Obs} \\
                \hline
                \textbf{$I$}  &  12.67415006 & 2.011584937 & 6.970118575 &16.65631825 &  102 \\
                \hline
                \textbf{$C_w$}  &  0.030206  &  0.011425 & 0.001864 & 0.065984 & 102 \\
                \hline
                \textbf{$C_I$} & 0.073909 & 0.022539 & 0.032302 & 0.134383 & 102 \\
                \hline
                \textbf{$C_O$} & 0.077249 & 0.022527 & 0.032 & 0.137046 & 102 \\
                \hline
        \end{tabular}
        \caption{Statistical summary of the metrics}
\end{table}
As we have three different metrics for the centralization, we decided to apply the Principal Component Analysis (\textbf{PCA}) to reduce the dimensionality of the data. The Principal Components Analysis (PCA) aim to represent the variation in the variables of a dataset using a smaller number of "main components".\\
Before applying the PCA we computed the Factor Analysis with two metrics: the Kaiser-Meyer-Olkin measure and the Bartlett's test of sphericity. The first one is a measure of sampling adequacy and the second one is a measure of the correlation between the variables. \\
For the Kaiser-Meyer-Olkin measure we obtained a value of 0.57, which is an acceptable value, and for the Bartlett's test of sphericity we obtained a p-value $<$ 0.00, which is a very low value, so the test is significant.
After that we applied the PCA and we obtained a single value for the centralization metrics as we keep the eigenvector with the highest eigenvalue, which is the main component of this dataset. \\
To evaluate the performance of a team as it is very difficult to evaluate the defensive performance of a team, and as the passes is an offensive metric, we decided to use the goal scored as a metric to evaluate the performance of a team. \\
So, our final dataset is composed by: team, match, network intensity, centralization (by PCA), goal scored.
To test our whole-network hypothesis we computed the Pearson's Correlation matrix on the dataset and we obtained the following results (we show only the results between intensity and Centralization whith goal scored):
\begin{table}[H]
        \centering
        \begin{tabular}{|c|c|c|}
                \hline
                & \textbf{Goal Scored}  \\
                \hline
                $Intensity$  &  $0.157593$  \\
                \hline
                $Centralization$  &  $-0.001456$ \\
                \hline
        \end{tabular}
        \caption{Correlation between the metrics and the goal scored}
    \end{table}

To test our hypothesis we trained a simple linear regression and we analyzed the coefficients of the regression and we noticed that the coefficient of the network intensity is positive and the coefficient of the centralization is negative. This means that there is a positive correlation between the network intensity and the goal scored and a negative correlation between the centralization and the goal scored. \\

\begin{table}[H]
    \centering
    \begin{tabular}{|c|c|c|}
            \hline
            & \textbf{Coefficient} & \textbf{p-value} \\
            \hline
            $Intensity$  &  $0.10951$ & $0.00042991182$  \\
            \hline
            $Centralization$  &  $-0.059461$  &  $0.9923159219812717$ \\
            \hline
    \end{tabular}
    \caption{Linear regression results}
\end{table}

Analyzing the p-value of the regression we can say that the network intensity is statistically significant, but the centralization is not. So we can conclude that the first hypothesis is confirmed, but the second one is not despite the fact that the coefficient is negative. \\

\subsection{Qualitative Analysis}
\label{qualitative-analysis}
\subsubsection{Pattern Recognition}
Analyze the network structure of a team give us a lot of information about the team's style of play and the way the coach organize the team in a particular match. We made a graphical visualization for each of the 102 observation we made, using a python library called \emph{mlpsoccer} and \emph{matplotlib}. We computed the average position of each player in the starting XI and we used this information to plot the network. The dimension of a node is proportional to the number of passes made by the player and the dimension of a edge is proportional to the strenght of that tie. Analyze these visualization give us a lot of insight. One of the most beautiful example we obtained is the network of the Spain against the Sweden. \\
%TODO: Change 
\begin{figure}[H]
        \centering
        \includegraphics[width=0.8\textwidth]{../NoSubs/ImagesToRedo/Spain_Network_Spain_Sweden.png}
        \caption{Spain-Sweden}
        \label{fig: spain_sweden}
\end{figure}
As you can easily see this network is perfectly symmetric and this can make us recognize a pattern. This match is the most interesting example of the way the Spanish team plays, but starting from this, we can see that the Spanish team plays in a very similar way in almost all the matches. \\
\begin{figure}[H]
        \centering
        \subfloat[\centering Spain-Poland]{{\includegraphics[width=0.4\textwidth]{../NoSubs/ImagesToRedo/Spain_Network_Spain_Poland.png} }}%
        \qquad
        \subfloat[\centering Switzerland-Spain]{{\includegraphics[width=0.4\textwidth]{../NoSubs/ImagesToRedo/Spain_Network_Switzerland_Spain.png} }}%
        \label{fig: spain_pattern}%
\end{figure}

Other interesting example are the networks of Russia against Denmark and Poland against Spain. In these two matches the strongest tie between two players is a pass from the goalkeeper to the centre forward. This is due to the difference of quality between the two teams and the high pressing the two opponent team made during these matches, so the only playing possibility for Poland and Russia were to pass the ball back to the goalkeeper and he had the only choice to make a long pass toward the striker. \\


\begin{figure}[H]
        \centering
        \subfloat[\centering Poland in Spain-Poland]{{\includegraphics[width=0.4\textwidth]{../NoSubs/ImagesToRedo/Poland_Network_Spain_Poland.png} }}%
        \qquad
        \subfloat[\centering Russia in Russia-Denmark]{{\includegraphics[width=0.4\textwidth]{../NoSubs/ImagesToRedo/Russia_Network_Russia_Denmark.png} }}%
        \label{fig:example}%
\end{figure}

\subsubsection{Node Centrality}
For each match, for each team we computed a weighted in-degree node centrality for each player. We used the in-degree centrality beacuse we are interested in how much other player in a team look for a certain player during a match. Then we computed the average weighted centrality for each player for each team. In some cases the results match our expectations, in other cases not. \\
The most evident example in which the difference between the average weighted centrality and the weighted centrality of a player is very high is the case of the Croatia team. As soon as Luka Modric (Croatia's captain) is one of the most important players in the world, and moreover in the football history, we did expect that he is the most important node in Croatia's networks. This hypothesis is totally confirmed as Modric is the node with the highest centrality in three on four matches played by Croatia in the tournament. The only match in which Modric is not the most important node in the network is the one against the Spain, in which the Croatia has been eliminated from the tournament losing 3-5. The reason behind the fact that in this match Croatia's captain was not the most important node is related with the high pressing Spain midfielders done on him on that match. This is a fantastic explanation of how analyze the network structure of a team can be very useful to gain information about the style of play of opponent teams and to make countermeasures.  \\ 

\begin{table}[H]
    \centering
    \begin{tabular}{|c|c|c|c|}
            \hline
            \textbf{Opponent} &  \textbf{Highest centrality} &  \textbf{Second highest centrality} &  \textbf{Mean} \\
            \hline
            \textbf{Czech Republic} &  Modric: $0.1645$ & Lovren: $0.1348$ & $0.0909$ \\
            \hline
            \textbf{ Scotland} &  Modric: $0.1621$ & Brozovic: $0.1439$ & $0.0909$ \\
            \hline
            \textbf{England} &  Modric: $0.1643$ & Kovacic: $0.1388$ & $0.0909$ \\
            \hline
            \textbf{Spain} &  Brozovic: $0.1745$ & Modric: $0.1236$ & $0.0909$ \\
            \hline
            \textbf{Total Sum} &  Modric: $0.6145$ & Kovacic: $0.4960$ & $0.3636$ \\
            \hline
    \end{tabular}
    \caption{Croatia's node in-degree centrality}
\end{table}



We expect to find the same result for the Portugal, as Cristiano Ronaldo is the most important player in the team. This is not the case, as the average weighted centrality of Ronaldo is lower than the weighted centrality of the other players. This is due to the fact that Ronaldo as a striker is not involved in the build-up of the play. The top two players for node centrality have been Ruben Dias and Pepe, the two centre back. This is the case of the majority of the teams for example Spain, England, France, Sweden, Ukraine and so on. This is due to two main factors: the evolution of the style of play of football teams which prefers to keep the ball possession so involve a lot the centre back in the build up and the fact that more and more team are no longer using a striker who act as a pivot and prefers a more mobile striker. \\
An honour mention to the winner of the tournament: Italy. We did expect that Jorginho Frello would be the most important node in the Italian network, and dispite he is the most central node in just 2 matches on 7, he has the highest average centrality, confirming our hypothesis. It is important to note that in the most matches the two most important nodes for Italy have always been Jorginho Frello or Marco Verratti, two midfielders. One important exceptions is the match against Spain (one of the most difficult match in the tournament for Italy) in wich the most important node has been Giorgio Chiellini, the centre back, for the same reason as Croatia-Spain. \\

%%%%%%%%%%%%%%%%%
%What is the connection among the gathered data, the applied measures,
%and the properties found?

\subsection{Conclusion}
\label{conclusion}
During a football match, the players of a team are constantly in contact with each other. The main way in which the players interact with each other is through passes. These passes create a network, representing the way the players of a team interact and so, the way the team plays.\\
In this study we analyzed 102 networks, from 24 different national teams in the Euro 2020 tournament. We have chosen Euro 2020 for three main reasons: the dataset was open, it is the last international tournament played and, being an international tournament whit knockout games, the level of players and intensity of the matches is higher.\\
Using the network structure of a team we focused on two main aspects: the performance of a team and the style of play of a team, the first performed by a quantitative analysis of the network structure and the second by a qualitative analysis of the network structure.\\
This study has shown that the network structure of a team can be a very useful tool to understand the performance of a team. We started with two hypotheses to test:
\begin{itemize}
        \item Increasing the intensity of a network increase the team performance.
        \item Increasing the centralization of a network decrease the team performance. 
\end{itemize}
We have shown that both the hypotheses are true, but for the first hypothesis the results are statistically significant, while for the second hypothesis are not.\\
Another interesting perspective of this study is how the network structure of a team can be a useful tool to understand the style of play of a team. We showed that observing the structure of a network it is easily to recognize patterns of play, we showed the most evident one showing Spain's networks. After that we analyzed the in-degree centrality of nodes (players) in networks 
to understand which player is the most important in the network. We showed that the most evident case is the Croatia's one in which Modric is the most important node in all matches, except one: the match again Spain in which Spain's midfielders obstruct Croatia blocking Modric. This is the most important example we found on how analyze a network can lead to new tactics that could be decisive in a tournament like this. \\
In conclusion this study has shown how important analyze the structure of the network of a team could be important to modify the style of play of a team and to counter the style of play of the opponent team. We tried to show how much these type of analysis can be useful to maximize the results of a team in a world like football where the data analysis is still a growing field and a lot of teams do not apply this type of knowledge.\\

\subsubsection{Critique}
\label{critique}
In this study we would to repeat the results obtained by another study \cite{GRUND} to test if the two hypothesis are true in a different context like an international tournament. The main difference between a domestic league and an international tournament is the level of the players and the intensity of the matches. In a domestic league the level of the players is lower, and the matches are less intense because rarely (especially during the first month of the competition) a single match is decisive. This is confirmed by the fact that our data for network intensity are higher respect to the data of the other study. 
We think that the main problem of our study is related to the limited quantity of data. The Grund study has analyzed two seasons of a domestic league (38 matches for 20 teams) for a total of 1520 networks while the Euro 2020 is a summer tournament played in a period of time of 3/4 weeks. Moreover, being Euro 2020 a knockout tournament half of the teams are eliminated after just three games, and only two teams plays the full tournament (reaching the final) playing a total of 7 matches. This is the main reason why we have a limited quantity of data if we want to perform this type of analysis on an international tournament. \\
The quantity of data we had can explain why we have not found a statistical significant result for the second hypothesis. We think that if we had more data we could have found a statistical significant result. \\ 


%Do you think your work solves the problem presented above? To which
%extent (completely, what parts)? Why? What could you have done
%differently to answer your research problems (e.g., gather data with
%additional information, build your model differently, apply alternative
%measures)?

\bibliographystyle{plain}
\bibliography{bibliography}
\nocite{*}
\end{document}
